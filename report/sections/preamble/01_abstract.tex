%!TEX root = ../main.tex
\chapter*{Abstract}

The history of the past 30 years shows that cyber-attacks are here to stay, and the security industry offers a great deal of protection, but threats come in many shapes and forms. Virtually all the threats reside in the software-hardware area but one. This threat is the manipulation of people to act as a proxy for the threat actor and carry out malicious actions, process called phishing.

Phishing has made a name for itself across the years. By taking advantage of human nature or flaws, the malicious actor is able to bypass most of, if not all security protections. The ease of conducting such an attack compared to other has popularised the belief that "humans are the weakest link in the cyber-security chain". The work presented in this dissertation will explore a method to mitigate some of the responsability from the target of phishing. Furthermore it aims to do that while being lightweight and performant, but most importantly actionable with minimal technical literacy.
%[The text within the square brackets must be deleted along with the square brackets when finalising your own abstract. 

%The abstract for an undergraduate dissertation should be between 200 - 350 words.

%Arial, Normal, 11pt with 1.2 or 1.5 line spacing should be used. The text in this part has 1.5 line spacing.

%An abstract is a brief, accurate and comprehensive summary of the entire dissertation. It is the first thing to be read by your examiners to help them know the brief content of the dissertation. It also serves as a “sales pitch” to form the first impression of your work. 

%A good abstract should be accurate, self-contained, concise, specific and clear. A quick way to assess the quality of your abstract is to check whether it answers the questions why, how, what and so what.

%It is easier to write the Abstract the last.]
