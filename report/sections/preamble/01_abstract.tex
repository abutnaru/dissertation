%!TEX root = ../main.tex
\chapter*{Abstract}

Social engineering has maintained a special place in the threat landscape throughout the last thirty years. What is different about these attacks is that they can compromise a target individual or enterprise through social interaction alone. Moreover, it can also serve as an attack vector or complement other types of exploits.
At the same time, part of the security community alleviates their responsibility or avoids involvement by describing the human factor as inevitably the weakest link in the chain.

This dissertation aims to investigate the truth value of the statement above. It concentrates on the phishing subclass of social engineering and explores a method to mitigate some of the responsibility from the target. The work presented evaluates the most popular anti-phishing detection system and builds a solution to challenge its offerings. This solution effectively exceeds the target browser-embedded protection in both performance and simplicity. Furthermore, it is developed to fit the prerequisites of a production-ready browser anti-phishing detection system.
