%!TEX root = ../main.tex
\chapter{Conclusion}
This chapter summarises the work presented throughout this dissertation, progressing through the steps followed to achieve the result. The following subsection lays out the evaluation of the artefact and discusses it's achievents vis-a-vis the set of objectives. The final section proposes possible paths of furthering the work presented, expands on the artefact's limitations and possible improvements.

\section{Summary}
The work and presented throguhout this dissertation aimed to challenge and improve the status-quo of browser anti-phishing detection systems. In doing so the objectives have been set around studying the current situation of the most popular browser anti-phishing detection system, surpass it's effectiveness and offer an easy to operate interface with the artefact produced.

The evaluation of Google Safe Browsing has been performed both through a realistic scenario using browser automation and the public Google Safe Browsing API. This analysis has uncovered that feeding the same phishing URLs after a period of time in the detection system results in a lower detection rate in the latter case, thus showing that it's classification influences the lifespan of a phishing webpage. Besides this, the false-negatives rate is consistently low and the process of including new online and valid phishes is slow.

After the threshold is set, the solution is built and experimented with. The first aim of the solution design is to be lightweight and deliver predictions only when necessary, thus it includes a white list of top one million domains known to be benign. The second aim is the delivery of accurate predictions based only on the URL of the webpage. This has been done in line with the prerequisites of a browser anti-phishing detection system of delivering low false positives at the expense of false negatives.
Because of the only source of data is the URL, a great amount of research and attention went into building the feature set. The great synergy between them is reflected in the substantially better results than the set threshold in both train and test data.

The performance improvement phase outlines the design decision of prioritising a balanced, robust solution instead of a phishing-detection optimised one. This is ahoqn in the  80\% phishing detection rate on a new test set, while the accuracy reaches 99.29\%.

Finally, the artefact and it's effectiveness despite the reduced data input proves that there is room for improvement in the anti-phishing detection systems available for the public. While it manages to surpass both GSB and the solutions presented by \cite{INTELLIGENT_PHISHING_ANFIS}, is worth considering that the only source of information is the URL, while other solutions use features such as HTML tag analysis, external links analysis and so on. Thus, not only it fulfils the aim, but it does so in a lightweight manner.

\section{Objectives review}
The first objective set is to research and develop a mechanism for evaluating the accuracy of the phishing detection system embedded in popular browsers. To address this, a testing tool has been developed that evaluates the accuracy both in a realistic scenario through browser automation and by checking the URLs against Google Safe Browsing's API.

The second objective is to research and corroborate different anti-phishing methods described in the literature into a multi-layered phishing detection design. In doing so, the artefact doesn't only use features described by the literature, but slightly alters some and introduces some new ones. 

The third and last objective is to implement and calibrate a phishing detection system that improves the status quo and can be operated with minimal technical literacy. The artefact produced achives a performance far surpasing what Google safe browsing offers. Furthermore, the artefact surpasses most of the solutions presented by the literature.

\section{Reflection}
Throughout the work carried out to produce the report and artefact presented, the aim and objectives have slightly evolved as the view on the subject matured. This section will expand some of the changes made to the project proposal (WILL ADD APPENDIX REF) offering justifications for the decisions taken.

The proposal starts with presenting an overview of the project. It defines the problem to be addressed and introduces the project idea. While the goal of offering an anti-phishing detection system that requires minimal technical literacy to operate was not altered, some of the features differ from the ones pitched. The artefact to be produced is described as being capable of recognising algorithmically generated domain names, but studying phishing techniques uncovered that these are usually markers of malware command and control servers. The second feature pitched is the inclusion of a blacklist. Although useful as proved by the literature, the machine learning model exceeded the accuracy expectations and the inclusion of phishtank datasets in a blacklist would tamper with the results of the comparison between it and GSB.

After creating a low resolution image of the artefact's design, the proposal briefly describes the method of evaluation. Doing the appropriate research proved that the methodology proposed is rudimentary and inacurate. The design of the evaluation has been changed to include sandard metrics used in the performance assessment of models in data science.

\section{Future work}
There are a number of ways to improve the classifier. Future work on the artefact could address either the blindspots or performance improvements. Another area worth considering is User Experience (UX). This can be significantly improved by wrapping the artefact in a browser extension.

An example of a blindspot is the inability to detect man-in-the-middle attacks. A scenario where the legitimate domain name is resolved to an IP address under the control of a threat actor would go under the radar of the artefact. This is solved by implementing a key-value store of domain names and IP addresses. To improve access times, this store could be managed by an in-memory data structure store (e.g. Memcached, Redis). Although the implementation of such feature is trivial, it is considered out of the scope of objectives set in Chapter 1.

Another path of improvement could be reducing the time of whitelist checking by using low-level compiled programming languages (e.g. C/C++, Rust) and an appropriate data structure (e.g. Hashmap, Radix Tree etc.). 

To further improve the accuracy of predictions, same classification can be applied on all the URLs found in the HTML source code of the page, thus both expanding surface coverage and lowering the number of false-negatives.

Finally, another area of possible improvement is inter-feature correlations and rule mining. \cite{INTELLIGENT_RULE_MINING} offers a detailed description of how such rule sets can be extracted and how they improve model efficiency.


% FEATURE TWEAKING + HYPERPARAMETER TUNING? Cannot detect when the legit website is highjacked
% 

\clearpage
\vspace*{\fill}
\begin{center}
\begin{minipage}{.6\textwidth}
\centering 
Report word count:

Artefact word count: (equivalent)
\end{minipage}
\end{center}
\vfill
\clearpage


\iffalse
The Conclusions chapter marks the end of the project report and it is a summary which brings together many of the points that you have made in other chapters, especially in the previous chapter. It is usually 2 – 3 pages long with three sections:
\begin{enumerate}
	\item Summary: summarise what you have achieved and restate the main results
	\item Evaluation: evaluate what you have achieved and how well you have met the objectives
	\item Future work: explain any limitations and how things might be improved.]
\end{enumerate}

\noindent
[Word count should be included at the end of the last section. Please look at section 5.2 and 5.3 of the Project Handbook for word count policy.]
\newline
Word count (main body of the report): 
\\
Word count (artefact): 

\fi